%% LyX 2.1.4 created this file.  For more info, see http://www.lyx.org/.
%% Do not edit unless you really know what you are doing.
\documentclass[english]{article}
\usepackage[T1]{fontenc}
\usepackage[latin9]{inputenc}
\usepackage{geometry}
\geometry{verbose,tmargin=1in,bmargin=1in,lmargin=1in,rmargin=1in,headheight=1in,headsep=1in,footskip=0.5in}
\usepackage{babel}
\usepackage{graphicx}
\usepackage{algorithm}
\usepackage{algpseudocode}
\usepackage{pifont}
\newif\ifdraft
\drafttrue
\ifdraft
\usepackage{xcolor}
\definecolor{ocolor}{rgb}{1,0,0.4}
\newcommand{\onote}[1]{ {\textcolor{ocolor} { (***Ole: #1) }}}
\newcommand{\terminology}[1]{ {\textcolor{red} {(Terminology used: \textbf{#1}) }}}
\newcommand{\owave}[1]{ {\cyanuwave{#1}}}
\newcommand{\jwave}[1]{ {\reduwave{#1}}}
\newcommand{\alwave}[1]{ {\blueuwave{#1}}}
\newcommand{\jhanote}[1]{ {\textcolor{red} { ***shantenu: #1 }}}
\newcommand{\alnote}[1]{ {\textcolor{blue} { ***andreL: #1 }}}
\newcommand{\amnote}[1]{ {\textcolor{blue} { ***andreM: #1 }}}
\newcommand{\smnote}[1]{ {\textcolor{brown} { ***sharath: #1 }}}
\newcommand{\pmnote}[1]{ {\textcolor{brown} { ***Pradeep: #1 }}}
\newcommand{\msnote}[1]{ {\textcolor{cyan} { ***mark: #1 }}}
\newcommand{\mrnote}[1]{ {\textcolor{purple} { ***melissa: #1 }}}
\definecolor{orange}{rgb}{1,.5,0}
\newcommand{\aznote}[1]{ {\textcolor{orange} { ***ashley: #1 }}}
\definecolor{dandelion}{cmyk}{0,0.29,0.84,0}
\newcommand{\mtnote}[1]{ {\textcolor{dandelion} { ***matteo: #1 }}}
\newcommand{\gpnote}[1]{{\textcolor{green} {***giannis: #1}}}
\newcommand{\note}[1]{ {\textcolor{magenta} { ***Note: #1 }}}
\else
\newcommand{\onote}[1]{}
\newcommand{\terminology}[1]{}
\newcommand{\owave}[1]{#1}
\newcommand{\jwave}[1]{#1}
\newcommand{\alnote}[1]{}
\newcommand{\amnote}[1]{}
\newcommand{\athotanote}[1]{}
\newcommand{\smnote}[1]{}
\newcommand{\pmnote}[1]{}
\newcommand{\jhanote}[1]{}
\newcommand{\msnote}[1]{}
\newcommand{\mrnote}[1]{}
\newcommand{\aznote}[1]{}
\newcommand{\mtnote}[1]{}
\newcommand{\note}[1]{}
\fi
\begin{document}

\title{Experiments Report}

\maketitle

\begin{abstract}
This report will include the discussion for the experiments. The experiments section will have
data plotting and an initial analysis (model and discussion) based on the developed understanding. 
A Q \& A subsection will follow after the discussion. I will add questions there that still need answering.
It would be nice if others contributed with questions!
\end{abstract}


\section{PSA: Hausdorff Distance}

The PSA Hausdorff Distance execution is happening upon a set of trajectories. It is similar to an
outer product multiplication between two vectors where the calculation instead of the multiplication
is the Hausdorff distance of two trajectories. The algorithm for this analysis is

Let $dH$ be the time to calculate the Hausdorff distance between
two trajectories. The following algorithm describes it in pseudocode.
The description will help the following analysis

\begin{algorithm}
\begin{algorithmic}[1]
\Procedure{HausdorffDistance}{$T_1,T_2$}\Comment{$T_1$ and $T_2$ are a set of 3D points}
\For{\forall $t_1$ in $T_1$}
\For{\forall$t_2$ in $T_2$}
\State \texttt{Append in $D_1$ calculated d($t_1, t_2$)}
\EndFor
\State\texttt{$D_t_1$ append max($D_1$)}
\EndFor
\State $N_1$ = min($D_t_1$)
\For{\forall $t_2$ in $T_2$}
\For{\forall$t_1$ in $T_1$}
\State \texttt{Append in $D_2$ calculated d($t_2, t_1$)}
\EndFor
\State\texttt{$D_t_2$ append max($D_2$)}
\EndFor
\State $N_2$ = min($D_t_2$)
\State \textbf{return} max($N_1$,$N_2$)
\EndProcedure
\end{algorithmic}
\end{algorithm}


%\begin{algorithm}
%Let \$T\_1\$ and \$T\_2\$ be the two trajectories
%\For{\forall t\_1\textbackslash{}in T\_1\$}
%\&nbsp;\&nbsp;For \$\textbackslash{}forall t\_2\textbackslash{}in
%T\_2\$ do:
%\&nbsp;\&nbsp;\&nbsp;\&nbsp;Append in \$D\_1\$ d(\$t\_1, t\_2\$)
%\&nbsp;\&nbsp;EndDo
%\&nbsp;\&nbsp;D = max(\$D\_1\$)
%EndFor
%\$N\_1\$=max(D)
%For \$\textbackslash{}forall t\_2\textbackslash{}in T\_2\$ do:
%\&nbsp;\&nbsp;For \$\textbackslash{}forall t\_1\textbackslash{}in
%T\_1\$ do:
%\&nbsp;\&nbsp;\&nbsp;\&nbsp;Append in \$D\_2\$ d(\$t\_2, t\_1\$);
%\&nbsp;\&nbsp;EndDo
%\&nbsp;\&nbsp;D = max(\$D\_2\$)
%EndFor
%\$N\_2\$=max(D)
%Return max(\$N\_1, N\_2\$)
%\caption{hello}
%\end{algorithm}

\section{Leaflet Finder}
There are two basic algorithms that are used in the task level parallel
and Spark implementation of leaflet finder. The first part of the
algorithm is the calculation of the adjanceny matrix and the second
the calculation of te connected components. The adjacency matrix is
calculated in different tasks. Each task is responsible for calculating
a submatrix. As a result each task executes $\Big(\frac{n}{k}\Big)^{2}$,
where $n$ is the total number of atoms and $k$ is the number of
equally distributed parts of the dataset. The connected components
of the networkx python package uses BFS from every node that is not
visited, thus the complexity is $O\Big(E^{2}+EV\Big)$ with $E$ the
number of nodes and $V$ the number of edges in the graph. Thus, the
execution time of each part is 
\begin{eqnarray}
ET_{AM} & = & \alpha_{AM}\Big(\frac{n}{k}\Big)^{2}+\beta_{AM}\Big(\frac{k(k+1)}{2}\Big)\label{eq:1}\\
ET_{CC} & = & \alpha_{CC}\Big(n^{2}+nV\Big)+\beta_{CC}\label{eq:2}
\end{eqnarray}
The term $\alpha$ is the multiplier of the execution time of each
task and $\beta$ is the multiplier for the communication overhead
of each backend. Communication overhead is considered the time needed
to describe and schedule each task as well as any time needed to gather
its results. Pointer $AM$ represents the Adjacency Matrix calculation
anc $CC$ the connected components calculation.\\
In our case, $n$ is145746 atoms, $V$ is the number of edges in the
graph (855192 in total). Thus the total execution time for the Leaflet
Finder is
\begin{equation}
ET_{LF}=ET_{AM}+ET_{CC}+\gamma\label{eq:3}
\end{equation}
\\
Term $\gamma$ in \ref{eq:3} is to account any extra time added by each backend
for starting and finalizing the execution of the application. 

In order to find the values of $\alpha_{AM},\beta_{AM},\alpha_{CC},\beta_{CC}$ and $\gamma$ we need to
solve a 5 by 5 linear system with form $b = Ax$, where $b$ the vector with the execution time, $A$ is a
5 b 5 matrix with the following values

\[
A=\left[\begin{array}{ccccc}
24291^2 & 21 & (145746^2 + 145746*855192)  & 1 & 1\\
16194^2 & 45 & (145746^2 + 145746*855192)  & 1 & 1\\
8097^2 & 171 & (145746^2 + 145746*855192)  & 1 & 1\\
5398^2 & 378 & (145746^2 + 145746*855192)  & 1 & 1\\
2699^2 & 1485 & (145746^2 + 145746*855192)  & 1 & 1\\
\end{array}\right]
\]

$x$ is 

\[
x=\left[\begin{array}{c}
\alpha_{AM}\\
\beta_{AM}\\
\alpha_{CC}\\
\beta_{CC}\\
\gamma\\
\end{array}\right]
\]

When this system is solved for RP-Spark execution on Comet the result is
\[\begin{array}{c}
\alpha_{AM}=5.81581051e-06, \beta_{AM}=-4.45020671e-02, \\
\alpha_{CC}=8.47564591e+06, \beta_{CC}=3.62162810e+17, \gamma=-1.59861300e+18
\end{array}\]


\section{CPPTraj RMSD}

\end{document}
